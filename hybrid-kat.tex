\documentclass[envcountsames]{llncs}

\usepackage{isabelle,isabellesym}
\usepackage{graphicx}
\usepackage{amsmath} 
\usepackage{amsfonts}
\usepackage{amssymb}
\usepackage{amscd}
\usepackage{latexsym}
\usepackage{mathrsfs}
\usepackage{stmaryrd}
\usepackage{mathpartir} % inferrules
%\usepackage{enumerate}
\usepackage{hyperref}
\usepackage{color}
\usepackage{tikz}
\usepackage{tikz-cd}
\usepackage[colorinlistoftodos]{todonotes}

\newcommand{\IF}[3]{\mathbf{if}\ #1\ \mathbf{then}\ #2\ \mathbf{else}\ #3}
\newcommand{\WHILE}[2]{\mathbf{while}\ #1\ \mathbf{do}\ #2}
\newcommand{\WHILEI}[3]{\mathbf{while}\ #1\ \mathbf{inv}\ #2\ \mathbf{do}\ #3}
\newcommand{\sskip}{\mathit{skip}}
\newcommand{\MKA}{\mathsf{MKA}}
\newcommand{\KAT}{\mathsf{KAT}}
\newcommand{\rKAT}{\mathsf{rKAT}}
\newcommand{\PDL}{\mathsf{PDL}}
\newcommand{\dL}{\mathsf{d}\mathcal{L}}
\newcommand{\flow}{\varphi}
\newcommand{\orbit}{\gamma^\varphi}
\newcommand{\lipschitz}{\ell}
\newcommand{\Pow}{\mathcal{P}}
\newcommand{\id}{\mathit{id}}
\newcommand{\Id}{\mathit{Id}}
\newcommand{\reals}{\mathbb{R}}
\newcommand{\bools}{\mathbb{B}}
\newcommand{\true}{\top}
\newcommand{\dLprog}{\Pi}
\newcommand{\ad}{\mathit{ad}}
\newcommand{\ar}{\mathit{ar}}
\newcommand{\wlp}{\mathit{wlp}}
\newcommand{\Set}{\mathbf{Set}}
\newcommand{\Rel}{\mathbf{Rel}}
\newcommand{\Sols}{\mathop{\mathsf{Sols}}}
\newcommand{\sta}{\mathsf{Sta}}
\newcommand{\rel}{\mathsf{Rel}}
\newcommand{\inv}{\mathsf{Inv}}

%\newcommand{\guards}[4]{(#1\circ #2)[#3,#4]}
\newcommand{\guards}[3]{#1\mathrel{\triangleright_#3} #2}
\newcommand{\gorbit}[1]{\gamma^#1_G}
\newcommand{\dinvar}[2]{( #1\hbox{ }\mathsf{invariant}\hbox{ }#2)}
%

\definecolor{scolor}{rgb}{1,0.5,0.5}
\definecolor{jcolor}{cmyk}{1,0,1,0}
\definecolor{gcolor}{cmyk}{1,0,0,0}


\newcommand\notein[3]{\todo[inline,linecolor=orange!80!black,backgroundcolor=#2!20]{#1: #3}%yellow!50 
}
\newcommand{\sfin}[1]{\notein{{\bf SF}}{scolor}{#1}}
\newcommand{\jin}[1]{\notein{{\bf JHM}}{jcolor}{#1}}
\newcommand{\gin}[1]{\notein{{\bf GS}}{gcolor}{#1}}

\urlstyle{rm}
\isabellestyle{it}

\begin{document}

\title{Hoare Logics and Refinement Calculi\\ for Hybrid Systems Verification} \titlerunning{Hoare Logics and Refinement Calculi for Hybrid Systems Verification}

\author{Simon Foster$^1$ \and Jonathan Juli\'an Huerta y Munive$^2$ \and Georg Struth$^2$} \authorrunning{Foster, Huerta y Munive and Struth}

\institute{University of York, UK \and University of Sheffield, UK}

\maketitle

\begin{abstract} 
  We present new Hoare logics and refinement calculi for the
  verification of hybrid programs in the style of differential dynamic
  logic.
\end{abstract}


%%%%%%%%%%%%%%%%%%%%%%%%%%%%%%%%%%%%%%%%%%%%%%%%%%


\section{Introduction}\label{sec:introduction}

Research question: We already have a predicate transformer calculus,
for instance based on $\MKA$. We know from~\cite{GomesS16} that we can
specialise it to a Hoare logic and a Morgan-style refinement
calculus. But how can we derive standalone Isabelle components based
on $\KAT$. 

technical contributions:
\begin{itemize}
\item first Hoare logic for $\dL$-style hybrid programs,
\item first Morgan-style refinement calculus for these,
\item Isabell verification components for both:
  \begin{itemize}
  \item post-hoc verification and refinement by supplying flows for
    Lipschitz continuous vector fields,
\item post-hoc verification and refinement with invariants for continuous vector
  fields,
\item post-hoc verification and refinement components for flow based
  hybrid programs,
  \end{itemize}
\item verification examples that demonstrate approach at work.
\end{itemize}

conceptual contributions: simple conceptual link between $\KAT$,
$\rKAT$, hybrid systems verification and standard verification
approaches beyond predicate transformers and $\dL$, now
domain-specific inference rules beyond assignment rules and refinement
laws for basic commands. 

Simplicity shows up in complexity and expressivity: equational theory
of $\KAT$ is PSPACE complete, that of $\MKA$, which underlies $\dL$, is
EXPTIME complete~\cite{MollerS06}. Weakest liberal preconditions,
which are the working horses of $\dL$, cannot be expressed in $\KAT$~\cite{Struth16}.

%%%%%%%%%%%%%%%%%%%%%%%%%%%%%%%%%%%%%%%%%%%%%%%%%%

\section{Kleene Algebra with Tests}\label{sec:kat} 

A \emph{Kleene algebra} is a structure $(K,+,\cdot,0,1,^\ast)$, where
$(S,+,\cdot,0,1)$ is a semiring with idempotent addition and the
Kleene star $(-)^\ast:K\to K$ satisfies, for all
$\alpha,\beta,\gamma\in K$, the axioms
\begin{align*} 
1+\alpha\cdot\alpha^\ast &\le \alpha^\ast, \qquad
  \gamma+\alpha\cdot
                                          \beta\le \beta\rightarrow \alpha^\ast \cdot \gamma\le \beta,\\
  1+\alpha^\ast\cdot\alpha &\le \alpha^\ast, \qquad \gamma+\beta\cdot
                             \alpha\le \beta\rightarrow \gamma\cdot
                             \alpha^\ast \le \beta.
\end{align*}
In these axioms, the ordering $\le$ on $K$ is defined by
$x\le y\leftrightarrow x+y=y$, as idempotent semirings are 
semilattices. We often write $\alpha\beta$ instead of
$\alpha\cdot\beta$.

A \emph{Kleene algebra with tests}~\cite{Kozen97} ($\KAT$) is a structure
$(K,B,+,\cdot,0,1,^\ast,\neg)$ where $(B,+,\cdot,0,1,\neg)$ is a
boolean algebra with join $+$, meet $\cdot$, complementation $\neg$,
least element $0$ and greatest element $1$, $B\subseteq K$, and
$(K,+,\cdot,0,1,^\ast)$ is a Kleene algebra. We henceforth write
$p,q,r,\dots$ for elements of $B$. 

Elements of $K$ represent programs; those of $B$ tests, assertions or
propositions.  The operation $\cdot$ models the sequential composition
of programs, $+$ their nondeterministic choice, $^\ast$ their finite
unbounded iteration. Program $0$ aborts and $1$ skips.  Tests are
meant to hold in some states of a program and fail in others;
$p\alpha$ ($\alpha p$) restricts the execution of program $\alpha$ in
its input (output) to those states where test $p$ holds.

Binary relations of type $\Pow\, (S\times S)$ form
$\KAT$s~\cite{Kozen97} when $\cdot$ is interpreted as relational
composition, $+$ as relational union, $^\ast$ as reflexive-transitive
closure and the elements of $B$ as subidentities---relations below the
relational unit. This grounds $\KAT$ within standard relational
imperative program semantics. Yet we prefer the isomorphic
representation as nondeterministic functions or \emph{state
  transformers} of type $S\to \Pow\, S$.  Composition $\cdot$ is then
interpreted as Kleisli composition
\begin{equation*} 
(f\circ_K g)\, x = \bigcup\{g\, y\mid y \in f\ x \}, 
\end{equation*} 
$0$ as $\lambda x.\ 0$ and $1$ as $\eta_S = \{-\}$.  Stars
$f^{\ast}\, x = \bigcup_{i\in\mathbb{N}} f^i\, x$ are defined with
  respect to Kleisli composition with $f^{0} = \eta_S$ and
  $f^{n+1} = f \circ_K f^{n}$. The boolean algebra of tests has
  carrier set $B_S=\{f:S\to \Pow\, S \mid f\le \eta_S\}$, where the
  order on function has been extended pointwise,  and complementation
  is given by
  \begin{equation*} 
    \overline{f}\, x =
  \begin{cases}
    \eta_S\, x, & \text{ if } f\, x = \emptyset,\\
\emptyset, & \text{ otherwise}.
  \end{cases}
\end{equation*}
Henceforth we freely identify predicates, sets and state 
transformers below $\eta_S$: 
$P\cong \{s\mid P\, s\}\cong \lambda s.\ \{x\mid x=s \land P\, s\}$.

\begin{proposition}\label{P:kleisli-ka}
$\sta\, S = ((\Pow\, S)^S,B_S,\cup,\circ_K,\lambda x.\
  \emptyset, \eta_S,(-)^{\ast},\overline{(-)})$
  forms a $\KAT$, the \emph{full state transformer $\KAT$} over the
  set $S$.
\end{proposition}
A \emph{state transformer $\KAT$} over $S$ is any subalgebra of
$\sta\, S$. 

We have already formalised Kleene algebras with and without tests at
type classes in Isabelle~\cite{afp:ka,afp:kat}.  As Isabelle's type
classes allow only one type parameter, we are using a non-standard
approach that expands a Kleene algebra $K$ by an \emph{antitest}
function $n:K\to K$ from which a \emph{test} function $t:K\to K$ is
defined as $t=n^2$. Then $K_t = \{\alpha \mid t\, \alpha = \alpha\}$
forms a boolean algebra in which $n$ acts as test complementation. It
can be used in place of the boolean algebra $B$.  A formalisation of
the relational model of $\KAT$ can be found in~\cite{afp:kat}; that of
the state transformer model is a contribution to this article.


%%%%%%%%%%%%%%%%%%%%%%%%%%%%%%%%%%%%%%%%%%%%%%%%%%

\section{Propositional Hoare Logic and Invariants}\label{sec:hl-invariants}

It is well known that $\KAT$ provides an algebraic semantics for while
programs,
\begin{align*}
  \IF{p}{\alpha}{\beta} = p\cdot \alpha + \bar p \cdot
  \beta\qquad\text{ and }\qquad
\WHILE{p}{\alpha} = (p\cdot \alpha)^\ast \cdot \bar p,
\end{align*}
captures validity of Hoare triples in
a partial correctness semantics,
\begin{equation*}
  \{p\}\, \alpha\, \{q\} \leftrightarrow p\alpha\neg q = 0,
\end{equation*}
or equivalently by $p\alpha\le \alpha q$ or $p\alpha = p\alpha q$, and
allows deriving the rules of \emph{propositional Hoare
  logic}~\cite{Kozen00}---disregarding assignments---which we use for
verification condition generation:
\begin{align}
  &\{p\}\, \sskip\, \{p\}, \label{eq:h-skip}\tag{h-skip}\\
  p\le p' \land \{p'\}\, \alpha\, \{q'\} \land q'\le q\ \rightarrow\ &
                                                                       \{p\}\,
                                                                       \alpha\,
                                                                       \{q\},\label{eq:h-cons}\tag{h-cons}\\
  \{p\}\, \alpha\, \{r\} \land \{r\}\, \beta\, \{q\}\ \rightarrow\
  &\{p\}\, \alpha\beta\, \{q\},\label{eq:h-seq}\tag{h-seq}\\
  \{tp\}\, \alpha\, \{q\}\land \{\neg tp\}\, \beta\, \{q\}\
  \rightarrow\ & \{p\}\, \IF{t}{\alpha}{\beta}\, \{q\},\label{eq:h-cond}\tag{h-cond}\\
  \{tp\}\, \alpha\, \{p\}\ \rightarrow\ & \{p\}\, \WHILE{t}{\alpha}\, \{\neg tp\}.\label{eq:h-while}\tag{h-while}
\end{align}
Specific rules for commands with invariant assertions
$\alpha\ \mathbf{inv}\ i$ are easily derivable (operationally,
$\alpha\, \mathbf{inv}\, i = \alpha$).  We define an \emph{invariant}
of an element $\alpha\in K$ as an element $i\in B$ that satisfies
\begin{equation*}
  \{i\}\, \alpha\, \{i\}.
\end{equation*}
Then, with $\mathbf{loop}\, \alpha$ as syntactic sugar for $\alpha^\ast$,
\begin{align}
  p\le i \land \{i\}\, \alpha\, \{i\}\land i\le q\ \rightarrow\
  &\{p\}\, \alpha\, \{q\},\label{eq:h-inv}\tag{h-inv}\\
  p \le i \wedge \{it\}\, \alpha\, \{i\} \wedge \neg t p\le q\
  \rightarrow \ & \{p\}\, \WHILEI{t}{i}{\alpha}\,  \{q\},\label{eq:h-while-inv}\tag{h-while-inv}\\
   p\le i \land \{i\}\, \alpha\, \{i\}\land i\le q\ \rightarrow\ &
                                                                   \{p\}\, \mathbf{loop}\, \alpha\,
    \mathbf{inv}\, i\, \{q\}. \label{eq:h-loop-inv}\tag{h-loop-inv}
\end{align}
We use (\ref{eq:h-inv}), which is a special case of (\ref{eq:h-cons}),
to reason about differential invariants for continuous evolutions of
hybrid systems in Section~\ref{sec:hoare-inv}, \ref{sec:refine} and
\ref{sec:from-flows}. Rule (\ref{eq:h-while-inv}) is standard for
reasoning with invariants for while loops; (\ref{eq:h-loop-inv})
specific to loops of hybrid programs---see
Section~\ref{sec:sta-hybrid}.

Generic rules for propositional Hoare logic in Isabelle have been
derived for Kleene algebra in~\cite{afp:ka}; more specific ones for
$\KAT$ in~\cite{afp:kat,afp:vericomp}. Basic Isabelle infrastructure for
reasoning with general invariants has been developed for this
article. 


%%%%%%%%%%%%%%%%%%%%%%%%%%%%%%%%%%%%%%%%%%%%%%%%%%

\section{Predicate Transformer Semantics for Hybrid Programs}\label{sec:sta-hybrid}

Hybrid programs of differential dynamic logic ($\dL$)~\cite{Platzer10} are defined by the syntax
\begin{equation*}
\mathcal{C}\ ::= \ x:=e \mid x' = f \, \&\, G \mid ?P\mid \mathcal{C};\mathcal{C}\mid \mathcal{C}+\mathcal{C}\mid \mathcal{C}^*
\end{equation*}
that adds \emph{evolution commands} $x' = f \, \&\, G$ to the language
of $\KAT$---function $?(-)$ embeds tests explicitly into programs.
Evolution commands introduce a time independent vector field $f$ for
an autonomous system of ordinary differential equations
(ODEs)~\cite{Hirsch09,Teschl12} and a guard $G$, a predicate modelling
boundary conditions or similar restrictions of temporal
evolutions. Guards are also known as \emph{evolution domain
  restrictions}~\cite{DoyenFPP18}.

Formally, vector fields are functions of type $S\to S$ on some open set
$S\subseteq \reals^n$ and $n\in\mathbb{N}$---the state space of
the hybrid program. We assume that program variables range from $0$ to
$n-1$, or pick them from any set $V$ isomorphic to the finite ordinal
$n$. Program states in $S$ can thus be identified with hybrid program
stores modelled as functions from variables in $V$ to values in
$\reals$, as usual.

Geometrically, vector field $f$ assigns a vector to any point of the
state space
$S$. A solution to the \emph{initial value problem} (IVP) for the pair
$(f,s)$ and initial value $(0,s)\in T\times S$, where $T$ is an open
interval in $\reals$ containing $0$, is then a function $X:T\to S$
that satisfies $X'\, t = f\, (X\, t)$---an autonomous system of ODEs
in vector form---and $X\, 0 = s$. Solution
$X$ is thus a curve in $S$ through $s$, parametrised in $T$ and
tangential to $f$ at any point in $S$; it is called \emph{trajectory}
or \emph{integral curve} of $f$ at $s$ whenever it is uniquely
defined ~\cite{Hirsch09,Teschl12}.

For IVP $(f,s)$ with continuous vector field $f:S\to S$ and initial
state $s\in S$ we define the set of solutions on
$T$ as
\begin{equation*}
\Sols f\, T\, s = \{X \mid \forall t\in T.\  X'\, t = f\, (X\, t)\land X\, 0 = s\}.
\end{equation*}
Each solution $X$ is then continuously differentiable and thus
$f\circ X$ integrable in $T$.  For $X\in \Sols\, f\, T\, s$ and
$G:S\to\bools$, we further define the $G$-\emph{guarded orbit} of $X$
along $T$ in $s$~\cite{MuniveS19} with the help of the state transformer
$\gamma^X_G:S\to \Pow\, S$ as 
\begin{equation*}
\gamma^X_{G}\, s= \{X\, t\mid t\in T\land \forall \tau\in
{\downarrow}t.\ G\, (X\, \tau)\},
\end{equation*}
where ${\downarrow}t = \{t'\in T\mid t'\le t\}$, and the
$G$-\emph{guarded orbital} of $f$ along $T$ in $s$~\cite{MuniveS19}
via the state transformer $\gamma^f_G:S\to \Pow\, S$ as
\begin{equation*}
  \gamma^f_G\ s = \bigcup\{\gamma^X_G\, s\mid X\in \Sols\, f\, T\, s\}.
\end{equation*}

In applications, ${\downarrow}t$ is typically an interval
$[0,t]\subseteq T$.  Expanding definitions,
\begin{equation*}
\gamma^f_G\, s = \{X\, t \mid X\in \Sols\, f\, T\, s \land t\in T
\land \forall \tau\in{\downarrow}t.\ G\, (X\, \tau)\}.
\end{equation*}
If $\top$ either denotes the predicate that holds of all states in $S$
or the set $S$ itself, we simply write $\gamma^f$ instead of
$\gamma^f_\top$.

We use $\gamma^f_G$ to define the semantics of the evolution command
$x'= f\, \&\, G$~\cite{MuniveS19} for any continuous $f:S\to S$ and
$G:S\to \bools$ simply as
\begin{equation}
{(x'= f\, \&\, G)} = \gamma^f_G.\label{eq:st-evl}\tag{st-evl}
\end{equation}

Defining the state transformer semantics of assignments is
straightforward~\cite{MuniveS19}. First we define a state update
function $f_a:V\to (S \to E) \to S\to S$ as
\begin{equation*}
f_a\, x\, e\, s = s[x\mapsto e\, s],
\end{equation*}
where $f[a\mapsto b]$ updates $f:A\to B$ by associating $a\in A$ with
$b$ and every $y\neq a$ with $f\, y$.  The ``expression''
${e:S\to \reals}$ is thus evaluated in state $s$ to $e\, s$.  Then we
lift $f_a\, x\, e:S\to S$ to state transformer
$x:= e:S \to \Pow\, S$ using $\eta_S$:
\begin{equation}
  (x:= e) = \lambda s.\ \{f_a\, x\, e\, s\}.\label{eq:st-assgn}\tag{st-assgn}
\end{equation}

The development in this section has been formalised with
Isabelle~\cite{afp:hybrid}, both for a state transformer and a
relational semantics. An instance of the latter for particular vector
fields with unique solutions forms the standard semantics of
differential dynamic logic. Due to the connection to orbits or
orbitals, the state transformer semantics is arguably conceptually
simpler and more elegant.


%%%%%%%%%%%%%%%%%%%%%%%%%%%%%%%%%%%%%%%%%%%%%%%%%%


\section{Flow-Based Hoare Logic for Hybrid
  Programs}\label{sec:hoare-flow}

The assignment axiom of Hoare logic needs little explanation. Our
semantics allows us to derive a variant with update functions instead
of substitutions:
\begin{equation}
\{\lambda s.\ P\, s[x\mapsto e\, s]\}\,  x:=e\, \{P\}. \label{eq:h-assgn}\tag{h-assgn}
\end{equation}
Hence it remains to derive a rule for evolution commands.  We restrict
our attention to Lipschitz-continuous vector fields for which unique
solutions to IVPs are guaranteed by Picard-Lindel\"of's
theorem~\cite{Hirsch09,Teschl12}.  These are \emph{(local)\ flows}
$\flow:T\to S\to S$ and $X=\flow_s=\lambda t.\ \flow\, t\, s$ is the
trajectory at $s$. Guarded orbitals $\gamma^f_G$  then specialise to
\emph{guarded orbits}
\begin{equation*}
  \gamma^f_{G,U} = \{\flow_s\, t\mid t\in U\land \forall\tau \in
  {\downarrow}t.\ G\, (\flow_s\, t)\},
\end{equation*}
where $U\subseteq T$ is a time domain of interest, typically an
interval $[0,t]$ for some $t\in T$~\cite{MuniveS19}.  Accordingly,
(\ref{eq:st-evl}) becomes
\begin{equation}
  (x' = f\, \&\, G)= \gamma^f_{G,U}.\label{eq:st-evl-flow}\tag{st-evl-flow}
\end{equation}
The following Hoare-style rule for evolution commands is then
derivable.
\begin{equation}
\{\lambda s\in S.\forall t\in U.\ (\forall
\tau\in {\downarrow}t.\ G\, (\flow_s\, \tau)) \rightarrow P\,
(\flow_s\, t)\}\, x' = f\, \&\, G\, \{P\}. \label{eq:h-evl}\tag{h-evl}
\end{equation}

This finishes the derivation of rules for a Hoare logic for hybrid
programs---to our knowledge, the first Hoare logic for hybrid
programs. As usual, there is one rule per programming construct, so
that their recursive application generates proof obligations that are
entirely about data-level relationships---the discrete and continuous
evolution of the hybrid program store.

The Hoare-style rule~(\ref{eq:h-evl}) supports the following procedure for reasoning with an evolution command $x' = f\, \&\, G$ on a set $U$ in this calculus:
\begin{enumerate}
\item Check that vector field $f$ satisfies the conditions for Picard-Lindel\"of's theorem ($f$ is Lipschitz continuous and $S\subseteq\reals^n$ is open).
\item Supply a (local) flow $\flow$ for $f$ with open interval of existence $T$ around $0$.
\item Check that $\flow_s$ solves the IVP $(f,s)$ for each $s\in S$. That is, $\flow_s'\, t = f\, (\flow_s\, t)$,  $\flow_s\, 0 = s$, and $U\subseteq T$.
\item If successful, apply rule~(\ref{eq:h-evl}).
\end{enumerate}

\gin{add example}


%%%%%%%%%%%%%%%%%%%%%%%%%%%%%%%%%%%%%%%%%%%%%%%%%%

\section{Invariant-Based Hoare Logic for Hybrid Programs}\label{sec:hoare-inv}

Alternatively, we can reason with invariants for evolution commands instead of supplying flows to the rule~(\ref{eq:h-evl}). For this we follow~\cite{MuniveS19} in capturing the meaning of these invariants semantically. Our invariants generalise both $\dL$-style differential invariants~\cite{Platzer12} and \emph{invariant sets} for actions~\cite{Teschl12} in dynamical systems theory and (semi)group theory.

A predicate $I:S\to\bools$ is an \emph{invariant} of the continuous vector field $f:S\to S$ and guard $G:S\to\bools$ \emph{along} $T\subseteq \reals$ if
\begin{equation*}
\bigcup \Pow\, \gamma^f_G\, I\subseteq  I.
\end{equation*}
The operation $\bigcup\circ\Pow$ is known as the Kleisli extension in the powerset monad. The definition of invariance unfolds to 
\begin{equation*}
\forall s.\ I\, s \to (\forall X\in\Sols f\, T\, s.\forall t\in T.\ (\forall \tau\in {\downarrow}t.\ G\, (X\, \tau)) \to I\, (X\, t)).
\end{equation*}
For $G=\top$ we call $I$ an \emph{invariant} of $f$ along $T$.  Intuitively, invariants can be seen as sets of orbits.

Next we relate invariants for evolution commands with Hoare triples.
\begin{proposition}\label{P:inv-prop}
  Let $f:S\to S$ be continuous, $G:S\to\bools$ and
  $T\subseteq \reals$. Then $I$ is an invariant for $f$ and $G$ \emph{along} $T$ if and only if
\begin{equation*}
 \{I\}\, x' = f\, \&\, G\, \{I\}.
\end{equation*}
\end{proposition}

This proposition and the following variant of (\ref{eq:h-inv}) are helpful for verification condition generation:
\begin{align}
  P\le I \land \{I\}\, x' = f\, \&\, G\, \{I\}\land (I\cdot G)\le Q\ \rightarrow\
  &\{P\}\, x' = f\, \&\, G\, \{Q\},\label{eq:h-invg}\tag{h-invg}
\end{align}

Applying (\ref{eq:h-invg}) in the context of Proposition~\ref{P:inv-prop} then requires a proof for invariance.

\begin{proposition}\label{P:invrules}
  Let $f:S\to S$ be a continuous vector field, $\mu,\nu:S\to\reals$
  differentiable and $T\subseteq \reals$. 
\begin{enumerate}
\item If $(\mu\circ X)' =(\nu\circ X)'$ for all $X\in \Sols f\, T\, s$, then $\mu = \nu$ is a differential invariant for $f$ along $T$,
\item if $(\mu\circ X)'\, t\leq(\nu\circ X)'\, t$ when $t> 0$, and $(\mu\circ X)'\, t\geq(\nu\circ X)'\, t$ when $t< 0$, for all $X\in \Sols f\, T\, s$,
  then $\mu < \nu$ is an invariant for $f$ along $T$,
\item $\mu\neq \nu$ if and only if $\mu < \nu$ or $\nu < \mu$,
\item $\mu \not\le \nu$ if and only if $\nu < \mu$,
\jin{get Hoare triple version in Isabelle}
\item if $\{I_i\}\, x' = f\, \&\, G\, \{I_i\}$ for $i\in\{1,2\}$, then $\{I_1\land I_2\}\, x' = f\, \&\, G\, \{I_1\land I_2\}$ and $\{I_1\lor I_2\}\, x' = f\, \&\, G\, \{I_1\lor I_2\}$.
%\item if $I_1$ and $I_2$ are differential invariants for $f$ along $T$, then so are $\lambda s.\ I_1\, s \land I_2\, s$ and $\lambda s.\ I_1\, s \lor I_2\, s$.
\end{enumerate}
\end{proposition}

Condition $(1)$ follows from the well-known fact that two continuously differentiable are equal if they intersect at some point and their derivatives are equal.

Therefore, the rule (\ref{eq:h-invg}) and Propositions~\ref{P:inv-prop} and~\ref{P:invrules} yield the following procedure for proving the triple $\{P\}\, x' = f\, \&\, G\, \{Q\}$ using a invariant.
\begin{enumerate}
\item Check whether a candidate predicate $I$ is a differential invariant:
	\begin{enumerate}
	\item transform $I$ into negation normal form;
	\item if $I$ is complex, reduce it with Proposition~\ref{P:invrules} (3), (4) and (5);
	\item for atomic $I$, apply Proposition~\ref{P:invrules} (1) and (2);
	\end{enumerate}
(if successful,  $\{I\}\, x' = f\, \&\, G\, \{I\}$ holds by Proposition~\ref{P:inv-prop}(2))
\item if successful, prove $P\le I$ and $(I\cdot G)\le Q$ to apply rule (\ref{eq:h-invg}).
\end{enumerate}



%%%%%%%%%%%%%%%%%%%%%%%%%%%%%%%%%%%%%%%%%%%%%%%%%%

\section{Refinement Calculi for Hybrid Programs}\label{sec:refine}

A \emph{refinement Kleene algebra with tests}
($\rKAT$)~\cite{ArmstrongGS16}  is a $\KAT$
$(K,B)$ expanded by an operation $[-,-]:B\times B\to K$ that
satisfies, for all $\alpha \in K$ and $p,q\in B$, 
\begin{equation*}
  \{p\}\, \alpha\, \{q\} \leftrightarrow \alpha\le [p,q].
\end{equation*}
The element $[p,q]$ of $K$ is called \emph{specification
  statement}~\cite{Morgan94}. It satisfies 
\begin{equation*}
  \{p\}\, [p,q]\, \{q\}\qquad \text{ and }\qquad \{p\}\, \alpha\, \{q\} \rightarrow \alpha\le [p,q];
\end{equation*}
which make $[p,q]$ the greatest element of $K$ that satisfies the Hoare
triple with precondition $p$ and postcondition $q$.  Indeed, in
$\sta\, S$ and for $S\subseteq \reals^n$,
\begin{equation*}
  [P,Q] = \bigcup \left\{f:S\to \Pow\, S \mid \{P\}\, f\, \{Q\}\right\}.
\end{equation*}

The following Morgan-style refinement laws~\cite{Morgan94} are derivable in
$\rKAT$~\cite{ArmstrongGS16}. 
\begin{align}
  1 &\le [p,p],\label{eq:r-skip}\tag{r-skip}\\
[p',q'] &\le [p,q],\qquad \text{ if } p\le p'\text{ and } q'\le q,\label{eq:r-cons}\tag{r-cons}\\
\alpha & \le [0,1],\\
[1,0] &\le \alpha,\\
[p,r]\cdot [r,q] &\le [p,q],\label{eq:r-seq}\tag{r-seq}\\
\IF{t}{[tp,q]}{[\neg tp,q]} &\le [p,q],\label{eq:r-cond}\tag{r-cond}\\
 \WHILE{t}{[tp,p]} &\le [p,\neg tp]. \label{eq:r-while}\tag{r-while}
\end{align}

For invariants and loops, we obtain, in addition, 
\begin{align}
  [i,i] &\le [p,q],\qquad \text{ if } p\le i \le q,\label{eq:r-inv}\tag{r-inv}\\
\mathbf{loop} [i,i] &\le [i,i]. \label{eq:r-loop}\tag{r-loop}
\end{align}

In $\sta\, S$, moreover, the following assignments laws are
derivable~\cite{ArmstrongGS16}.
\begin{align}
 (x := e)  &\le  [\lambda s.\ P\, s[x\mapsto e\, s],P],\label{eq:r-assgn}\tag{r-assgn}\\
(x:= e) \cdot [\lambda s.\ P\, s[x\mapsto e\, s],P] &\le [P,P],\label{eq:r-assgn}\tag{r-assgnl}\\
[P,\lambda s.\ P\, s[x\mapsto e\, s]]\cdot (x:=e) &\le [P,P]. \label{eq:r-assgn}\tag{r-assgnf}
\end{align}
The second and third law are known as \emph{leading} and \emph{following}
law. They introduce an assignment before and after a block of code. 

As refinement law for evolution commands we obtain
\begin{equation}
x' = f\, \&\, G \le [\lambda s.\forall t\in U.\ (\forall
\tau\in {\downarrow}t.\ G\, (\flow_s\, \tau))\to P\, (\flow_s\, t),P]. \label{eq:r-evl}\tag{r-evl}
\end{equation}

\gin{is that correct? we also need to derive leading/following laws
  for evolution statements.}

These laws suffice for constructing hybrid programs from an initial
specification statement by step-wise refinement in an incremental and
compositional way in the style of Morgan. To our knowledge this is the
first refinement calculus for hybrid programs in this style.  A more
powerful refinement calculus based on predicate transformers in the
style of Back and von Wright~\cite{BackW98} has been developed
in~\cite{MuniveS19}, but applications remain to be explored.

\gin{do two examples; one for flows, one for invariants. One in one
  go, one compositionally.}

%%%%%%%%%%%%%%%%%%%%%%%%%%%%%%%%%%%%%%%%%%%%%%%%%%

\section{Starting from Flows}\label{sec:from-flows}

\gin{do components that use flows instead of vector fields in evolution commands}

%%%%%%%%%%%%%%%%%%%%%%%%%%%%%%%%%%%%%%%%%%%%%%%%%%

\section{Conclusion}\label{sec:conclusion}


%%%%%%%%%%%%%%%%%%%%%%%%%%%%%%%%%%%%%%%%%%%%%%%%%%

\bibliographystyle{abbrv}
\bibliography{hybrid-kat.bib}

\end{document}