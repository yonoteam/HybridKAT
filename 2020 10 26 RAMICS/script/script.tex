\documentclass[english,letterpaper,12pt]{article}

\usepackage{graphicx} % include graphics
\usepackage{amsmath} % official of American Mathematical Society
\usepackage{amssymb} % adds useful symbols like \Cap and \Cup
\usepackage{amsfonts} % includes fraktur and subscript bold
%\usepackage{amscd} % for rectangular arrow diagrams
\usepackage{mathrsfs} % allows fonts \mathscr
\usepackage{stmaryrd} % For doble brackets [[ ]]
\usepackage{mathpartir} % inferrules
\usepackage{color} % allows changin colors
\usepackage[colorinlistoftodos]{todonotes}
\usepackage{hyperref} % for hyperlinks in PDF, ensure it is last package


\newcommand{\dL}{\mathsf{d}\mathcal{L}}
\newcommand{\KA}{\mathsf{KA}}

\newcommand{\Pow}{\mathcal{P}}
\newcommand{\Sup}{\mathop{Sup}}
\newcommand{\id}{\mathit{id}}
\newcommand{\Id}{\mathit{Id}}
\newcommand{\reals}{\mathbb{R}}
\newcommand{\bools}{\mathbb{B}}

\newcommand{\cball}[2]{\overline{B_{#1}(#2)}}
\newcommand{\norm}[1]{\left\lVert #1\right\rVert}
\newcommand{\Sols}{\mathop{\mathsf{Sols}}}

\newcommand{\orbit}{\gamma^\varphi}
\newcommand{\gorbit}[1]{\gamma^#1_G}

\definecolor{gcolor}{cmyk}{1,0,0,0}
\newcommand\notein[3]{\todo[inline,linecolor=orange!80!black,backgroundcolor=#2!20]{#1: #3}}
\newcommand{\gin}[1]{\notein{{\bf GS}}{gcolor}{#1}}

\urlstyle{rm} %changes font of url's

\begin{document}
%
%\title{Affine Systems of ODEs in Isabelle/HOL for Hybrid-program Verification}
%
%\author{Jonathan Juli\'an Huerta y Munive}
%
%\maketitle

%%%%%%%%%%%%%%%%%%%%%%%%%%%%%%%%%%%%%%

IMPORTANT thing is to explain WHY? we do things

(context of research agenda)

\begin{itemize}
\item Slide 0: Explain title enthusiastically, don't read. 
\item Slide 1: Verification of Hybrid Systems
\begin{itemize}
\item VERY SIMPLE example
\item ``moving in a straight line at a constant speed $v_0$ to its docking station''
\item we model the behaviour of the spaceship as a hybrid program
\item we ``execute'' the differential equation after the control of the spaceship
\item we can place this hybrid program inside a PARTIAL correctness specification
\end{itemize}
\item Slide 2: Previous work
\begin{itemize}
\item in previous work we created FIRST verification components inside Isabelle/HOL for verification of hybrid programs
\item the rules for verification condition generation tackle the program structure recursively until we are left with a proof about properties of real numbers
\item because there is lack for proof support in Isabelle, to verify ODEs, users of our components need to follow this procedure
\item procedure is very IMPORTANT
\end{itemize}
\item Slide 3: Affine systems of ODEs
\begin{itemize}
\item going back to our example this procedure requires us to...
\item observe that we can rephrase the dynamics as a matrix
\item in fact, many systems in nature behave as affine systems
\end{itemize}
\item Slide 4: Main contributions
\begin{itemize}
\item NOT ONLY serves verification but it is a formalisation of undergraduate mathematics on its own 
\item GENERIC proof of existence and uniqueness
\item MEDIUM sized contribution to the AFP (40 pages of proofs)
\end{itemize}
\item Slide 5: Hybrid Programs: \dots to understand the need for this procedure, I want to deviate a bit and explain the behaviour of our verification components
\item Slide 6: What about ODEs?\dots however, in verification of hybrid programs, people like to add boundary conditions that restrict the evolution of the system
\item Slide 7: Semantics for ODEs
\begin{itemize}
\item guarded orbits are just traditional orbits whose initial segments are guarded by the boundary condition
\item we can derive in Isabelle one rule for verification condition generation per each program construct as in Hoare logic
\end{itemize}
\item Slide 8: Verification of Affine Systems
\item Slide 9: Formalising existence and uniqueness
\begin{itemize}
\item Isabelle works like a functional programming language
\item 10 pages of proofs equiv. to 600+ lines of code
\end{itemize}
\item Slide 10: Formalising solutions of affine systems (16 pages of proofs equiv. to 900+ lines of code)
\item Slide 11: verification example
\begin{itemize}
\item Very simple verification example, but more advanced examples can be found in the AFP and in the ARCH 2020 competition. 
\end{itemize}
\item Slide 12: handling exponentials
\item Slide 13: ceritfying a diagonalisation
\item Slide 14: conclusions
\end{itemize}


%%%%%%%%%%%%%%%%%%%%%%%%%%%%%%%%%%%%%%

Hello everyone, I am Jonathan Julian Huerta y Munive and I am finishing my PhD studies at the university of Sheffield. I came to talk to you about a formalisation of affine systems of ordinary differential equations that we did in Isabelle with the purpose of verifying hybrid programs.

To exemplify what I mean by hybrid program verification, I brought a small example. So imagine there is a spaceship moving at a constant speed $v_0$ in a straight line towards its docking station. The ship needs to stop exactly at a distance $d$ of its centre of mass and the current position of the tip of its nose is at $x_0$. The ship's computer determines that it needs to decelerate at this rate in order to stop just at the moment where its tip reaches the docking station. To avoid any catastrophic scenarios, we can model the ship's control with an assignment setting its current acceleration to the one determined by the computer. Then we can execute the differential equation of the dynamics of the ship after

% To exemplify what I mean by hybrid program verification, I brought a small example. So imagine there is a spaceship moving at a constant speed $v_0$ in a straight line towards its docking station. The ship needs to stop exactly at a distance $d$ of its centre of mass and the current position of the tip of its nose is at $x_0$. The ship's computer determines that it needs to decelerate at this rate in order to stop just at the moment where its tip reaches the docking station. To avoid any catastrophic scenarios, we can model the ship's control with an assignment setting its current acceleration to the one determined by the computer. Then we can execute the differential equation of the dynamics of the ship after

\end{document}
